\documentclass[12pt]{article}
\setlength\headheight{14.5pt}
\title{Homework}
\author{Frederick Robinson}
\date{16 November 2009}
\usepackage{amsfonts}
\usepackage{fancyhdr}
\pagestyle{fancyplain}

\begin{document}

\lhead{Frederick Robinson}
\rhead{Math 381: Fourier Analysis}

\maketitle

\setcounter{tocdepth}{2} 

\tableofcontents

\section{Chapter 4 Section 1}

\subsection{Problem 9}\label{4.1.9}
Show that the general solution of $\nabla^2\left[f(r)\right]=0$ is $f(r)=A+B/r$ for arbitrary constants $A, B$.

We have \cite[Page 237]{pinsky} by 4.1.5 that

\begin{equation}\label{laplacian}\nabla^2u=u_{rr}+\frac{2}{r}u_r+\frac{1}{r^2}(u_{\theta\theta}+\cot{\theta}u_\theta+\csc^2{\theta}u_{\varphi\varphi})\end{equation}

but, since $u(r)$ depends only on $r$ we have that

\[\nabla^2\left[f(r)\right]=0\]
\[\iff u_{rr}+\frac{2}{r}u_r+\frac{1}{r^2}(u_{\theta\theta}+\cot{\theta}u_\theta+\csc^2{\theta}u_{\varphi\varphi})=0\]
\[\iff f''(r)+\frac{2}{r}f'(r)=0\]

but this is just an ordinary differential equation, whose solution we know to be

\begin{equation}\label{homo}f(r)=A+\frac{B}{r}\end{equation}

So we have shown that the general solution of $\nabla^2\left[f(r)\right]=0$ is $f(r)=A+B/r$ for arbitrary constants $A, B$ as desired.

\subsection{Problem 11}

Solve the equation $\nabla^2\left[f(r)\right]=-1$ with the boundary conditions $f(a)=0$ and $f(0)$  finite.

From Exercise 10 \cite[Page 250]{pinsky} we know that the general solution for differential equations of this form is given by

\[f(r)=A+\frac{B}{r}-\frac{r^2}{6}\] for arbitrary constants $A, B$.

Now we apply our boundary conditions to determine $A, B$. Since $f(0)$ is finite we know that $B=0$. Thus,

\[f(a)=A+\frac{B}{a}-\frac{a^2}{6}=0\]
\[\Rightarrow 6Aa+6B-a^3=0\]
\[\Rightarrow 6A-a^2=0\]
\[\Rightarrow 6A=a^2\]
\[\Rightarrow A=\frac{a^2}{6}\]

Since we have determined $A=\frac{a^2}{6}$ and $B=0$ the particular solution of the equation $\nabla^2\left[f(r)\right]=-1$ with the boundary conditions $f(a)=0$ and $f(0)$  finite is given by:

\[f(r)=\frac{a^2}{6}-\frac{r^2}{6}\]
\[=\frac{a^2-r^2}{6}\]

\subsection{Problem 13}

Solve the equation $\nabla^2\left[f(r)\right]=-r^4$ with the boundary condition $f(a)=0$ and $f(0)$ finite.

We employ Equation \ref{laplacian} to reduce the general problem to that of an ordinary differential equation as in Problem 9 (\ref{4.1.9}).

\[\nabla^2\left[f(r)\right]=-r^4\]
\[\iff u_{rr}+\frac{2}{r}u_r+\frac{1}{r^2}(u_{\theta\theta}+\cot{\theta}u_\theta+\csc^2{\theta}u_{\varphi\varphi})=-r^4\]
\[\iff f''(r)+\frac{2}{r}f'(r)=-r^4\]

but as previously stated the solution to the corresponding homogeneous ODE is given by Equation \ref{homo}. To solve the current problem we need only guess a particular solution. Let's try 

\[f(r)=-\frac{1}{42}r^6\]

\[\Rightarrow f''(r)+\frac{2}{r}f'(r)=-r^4\] So this is indeed a solution to our equation. Thus the general solution to the ODE is given by a sum of the above and Equation \ref{homo}.

\[f(r)= A+\frac{B}{r} - \frac{1}{42}r^6\]

Now employing our boundary conditions we see that since $f(0)$ is finite $B=0$, and moreover 

\[f(a)=0\]
\[\Rightarrow f(a)=A-\frac{1}{42}a^6=0\]
\[\Rightarrow A=\frac{1}{42}a^6\]

So our final solution, including boundary conditions is given by

\[f(r)= \frac{1}{42}a^6 - \frac{1}{42}r^6\]

\[f(r)= \frac{1}{42}(a^6 - r^6)\]




\subsection{Problem 15}

Find the solution $u(r;t)$ of the heat equation $u_t = K \nabla^2u$, $-\infty<t<\infty$, in the sphere $0 \leq r <a$ satisfying the boundary condition $u(a;t) = A_1 \cos{\omega(t-t_0)}$ where $A_1, \omega$ and $t_0$ are positive constants.

We note that this problem has already been solved in more generality \cite[Page 239]{pinsky} and has solution

\[u(r;t)=\frac{a}{r}\sum^\infty_{n=-\infty}{\alpha_n e^{i \omega_n t}\frac{e^{c_n(1+i)r}-e^{-c_n (1+i) r}}{e^{c_n(1+i)a}-e^{-c_n (1+i) a}}}\] for $\omega_n=2\pi n/T$, $c_n=\sqrt{\pi n/KT}$

So all that remains is to specify the solution satisfying our boundary conditions by setting 

\[u_0(t)=\sum^\infty_{n=-\infty}\alpha_n e^{t\pi i n t/T}\] and computing the fourier coeffecients $\alpha_n$.

It is perhaps useful to first solve this problem for $t_0=0$.

In this case it is easy to see that since $u_0(t)=A_1\cos{\omega t}$ we have $\alpha_1=\alpha_{-1}=\frac{1}{2}A_1$ as in Example 4.1.2\cite[Page 239]{pinsky}. So, plugging in and taking the real part we see that the particular solution to this reduced problem is given by

\[u(r;t)=\frac{a A_1}{r}Re\ { e^{i \omega t}\frac{e^{c(1+i)r}-e^{-c (1+i) r}}{e^{c(1+i)a}-e^{-c (1+i) a}}}\] for $c=\sqrt{\omega/2K}$. Again, this solution is very similar to the one derived in Example 4.1.2\cite[Page 239]{pinsky}. 

However, the original problem is just a translation in time of the one which we have just solved.  Thus, its solution is given by 

\[u(r;t)=\frac{a A_1}{r}Re\ { e^{i \omega (t-t_0)}\frac{e^{c(1+i)r}-e^{-c (1+i) r}}{e^{c(1+i)a}-e^{-c (1+i) a}}}\] for $c=\sqrt{\omega/2K}$.




\subsection{Problem 17}

Find the solution $u(r;t)$ of the heat equation $u_t = K \nabla^2u+\sigma$ in the sphere $0 \leq r <a$ satisfying the boundary condition $u(a;t)=T_1$ and the initial condition $u(r;0)=T_2$. Use the five-stage method, and find the relaxation time.

Throughout we will follow the example of the book\cite[Page 242]{pinsky}.

First we will reduce this problem to a one dimensional one. Since the initial and boundary conditions are both independent of $(\theta, \varphi)$ we can reasonably assume that the solution is as well. Thus, we examine solutions of the form $u=u(r;t)$. Now we define a new function $w$ by $w(r;t)=ru(r;t)$. So, $w_t=ru_t$, $w_r=ru_r+u$, $w_{rr}=ru_{rr}+2u_r=r\nabla^2u$. Multiplying the problem for $u$ by $r$ to put it in terms of $w$ we get

\[ w_t=Kw_{rr}+\sigma r\]
\[ w(a;t)=aT_1\]
\[w(r;0)=rT_2\]
\[w(0;t)=0\]
for $0 \leq r <a$.

This one-dimensional boundary vale problem for $w$ can now be solved by the five-stage method outlined in the text\cite[Page 242]{pinsky}.

\subsubsection{Stage 1}

The steady-state equation is $KW_{rr}+\sigma r=0$ with the two boundary conditions at $r=0$, $r=a$. The general solution of this ordinary differential equation is just
\[W(r)=-\frac{\sigma r^3}{6K}+A+Br\]
for arbitrary constants $A, B$. The boundary condition $W(0)=0$ requires that $A=0$. To determine $B$ we take
\[W(a)=-\frac{\sigma a^3}{6K}+B a = a T_1\]
\[\Rightarrow -\frac{\sigma a^2}{6K}+B  =  T_1\]
\[\Rightarrow B  =  T_1+\frac{\sigma a^2}{6K}\]
substituting our constants into the general solution we get
\[W(r)=-\frac{\sigma r^3}{6K}+r\left(T_1+\frac{\sigma a^2}{6K}\right)\]


\subsubsection{Stage 2}

We use the steady-state solution to transform the problem. Letting $v(r;t)=w(r;t)-W(r)$, we have the equation for $v$:

\[v_t=Kv_{rr}\]
\[v(0;t)=0\]
\[v(a;t)=0\]
\[v(r;0)=\frac{\sigma r^3}{6K}-r\left(T_1-T_2+\frac{\sigma a^2}{6K}\right)\]

\subsubsection{Stage 3}
Now we determine the separated solutions to the problem for $v$. We write $v(r;t)=R(r)T(t)$ yielding the equations $T'+\lambda K T = 0$, $R''+\lambda R=0$ with boundary conditions $R(0)=0$, $R(a)=0$. The first equation can be solved with a constant by $T(t)=e^{-\lambda K t}$. The equation for $R(r)$ is a Sturm-Liouville eigenvalue problem,
\[R''+\lambda R=0 \]
\[R(0)=0\]
\[R(a)=0\]
which has solutions as in Example 1.6.1\cite[Page 85]{pinsky} given by 
\[\lambda_n=\left(\frac{n\pi}{a}\right)^2,\ \phi_n(x)=\sin{\frac{n\pi r}{a}}\ n=1,2,\dots\]

Now we can write the superposition of separated solutions as
\begin{equation}\label{super}v(r;t)=\sum_{n=1}^\infty A_n \sin{\frac{n\pi r}{a}} e^{-\lambda_n K t}\end{equation}

By Theorem 1.5\cite[Page 86]{pinsky}, the eigenfunctions must be orthogonal. Thus,

\[\int_0^a \sin{\frac{m\pi r}{a}} \sin{\frac{n\pi r}{a}} dr =0 \ n\neq m\]


The normalization can be computed as an integral

\[\int_0^a \sin^2{\frac{m\pi r}{a}}  dr = \frac{1}{2} \int_0^a \left(1- \cos{ \frac{2 m\pi r}{a}} \right) dr\]
\[= \left[\frac{r}{2}-\frac{a \sin{\left(\frac{2 m \pi  r}{a}\right)}}{4 m \pi }\right]_0^a \]
\[=\frac{a}{2} \left(1-\frac{ \sin{2 m \pi }}{2 m \pi }\right)\]

The Fourier Coefficients $A_n$ can be determined by setting $t=0$ in Equation \ref{super}, multiplying by $\sin{\frac{n\pi r}{a}}$ and integrating; that is

\[\int_0^a\left(\frac{\sigma r^3}{6K}-r\left(T_1-T_2+\frac{\sigma a^2}{6K} \right) \right) \sin{\frac{n\pi r}{a}}dr = A_n \int_0^a \sin^2{\frac{n\pi r}{a}}  dr \]
\[=\frac{A_n a}{2} \left(1-\frac{ \sin{2 n \pi }}{2 n \pi }\right)\]
\[\Rightarrow \frac{1}{3 K n^4 \pi ^4}a^2 (3 n \pi  \left(K n^2 \pi ^2 (T_1-T_2)+a^2 \sigma \right) \cos{n \pi }+\]
\[\left(3 K n^2 \pi ^2 (-T_1+T_2)+a^2 \left(-3+n^2 \pi ^2\right) \sigma \right) \sin{n \pi})\]

Since we are dealing only with $n \in \mathbb{N}$ we may replace $\sin{n\pi}$ with $0$ and $\cos{n\pi}$ with $(-1)^n$. Making this simplification we are left with

\[\frac{1}{3 K n^4 \pi ^4}a^2 (3 n \pi  \left(K n^2 \pi ^2 (T_1-T_2)+a^2 \sigma \right) (-1)^n=\frac{A_n a}{2} \]

Thus, the Fourier Coefficients $A_n$ are given by

\[A_n = \frac{2 (-1)^n a(T_1-T_2)}{n \pi }+\frac{2 (-1)^n a^3 \sigma }{k n^3 \pi ^3}\]


\subsubsection{Stage 4}

We have obtained the formal solution of the problem as 

\[ u(r;t)=U(r)+\frac{1}{r}\sum_{n=1}^\infty A_n \sin{\frac{n\pi r}{a}} e^{-\lambda_n K t}\]
with
\[U(r)=\frac{W(r)}{r}=T_1+\frac{\sigma}{6K}(a^2-r^2)\]
\[A_n = \frac{2 (-1)^n a(T_1-T_2)}{n \pi }+\frac{2 (-1)^n a^3 \sigma }{k n^3 \pi ^3}\]
\[\lambda_n=\left(\frac{n\pi}{a}\right)^2\]

\subsubsection{Stage 5/Relaxation Time}
When $t\to\infty$, the solution $u(r;t)$ tends to the steady-state solution $U(r)$. We use the method from Chapter 2 to estimate the rate of approach; thus using O from 2.2.3\cite[Page 113]{pinsky} we get
\[\frac{1}{r}\sum_{n=1}^\infty A_n \sin{\frac{n\pi r}{a}} e^{-\lambda_n K t}=O( e^{-a t})\ t\to\infty\]

Therefore $u(r;t)-U(r)=O( e^{-a t})$, $t\to\infty$. Finally we compute the relaxation time by noting that
\[ u(r;t)-U(r)= \frac{ A_1 \sin{\frac{\pi r}{a}}}{r} e^{-\lambda_1 K t}+O( e^{-a t})\ t\to\infty\]

If $A_1 \neq 0$ the relaxation time is given by 

\[\tau=\frac{1}{\lambda_1 K}=\frac{a^2}{\pi^2 K}\]



\section{Chapter 5 Section 1}
In the following problems we compute the Fourier Transform of the given function $f(x)$.
\subsection{Problem 1}

 \begin{displaymath}
   f(x) = \left\{
     \begin{array}{lr}
       1 & : x \in (-2,2)\\
       0 & : otherwise
     \end{array}
   \right.
	\end{displaymath} 


So we proceed employing the given definition for the Fourier Transform\cite[Page 278]{pinsky}:

$$F ( \mu ) = \frac{1}{2\pi} \int_{-\infty}^\infty f(x) e^{i \mu x} dx$$

Since the function $f$ is zero except on the interval $x \in (-2,2)$ we may restrict the limits of our integration to this region. 

Plugging in to the definition yields:

$$F(\mu)=\frac{1}{2\pi}\int_{-2}^2e^{i\mu x}dx$$

Now we evaluate the antiderivative to get

$$F(\mu)=\frac{1}{2\pi}\left[-\frac{i e^{i x \mu }}{\mu}\right]_{-2}^2$$

$$=\frac{1}{2\pi}\left(\frac{i e^{-2 i \mu }}{\mu }-\frac{i e^{2 i \mu }}{\mu }\right)$$

$$=\frac{2}{\pi\mu}\left(\frac{i e^{-2 i \mu }-i e^{2 i \mu }}{4 }\right)$$

$$=\frac{2}{\pi\mu}\left(\frac{e^{2 i \mu }- e^{-2 i \mu } }{4 i }\right)$$

Since the numerator is just a difference of two squares so we can express this as

$$\frac{2}{\pi\mu}\left(\frac{(e^{ i \mu }+ e^{- i \mu })(e^{ i \mu }- e^{- i \mu }) }{4 i }\right)$$

$$=\frac{2}{\pi\mu}\left(\frac{e^{ i \mu }+ e^{- i \mu }}{2 }\right)\left(\frac{e^{ i \mu }- e^{- i \mu } }{2i }\right)$$

$$=\frac{2}{\pi\mu}\cos{\mu}\sin{\mu}$$

So, the Fourier Transform of the given function is

$$F(\mu)=\frac{2}{\pi\mu}\cos{\mu}\sin{\mu}$$

\subsection{Problem 3}

 \begin{displaymath}
   f(x) = \left\{
     \begin{array}{lr}
       e^{-3x} & : x >0 \\
       e^{2x} & : x<0
     \end{array}
   \right.
	\end{displaymath} 

Again we use the definition for the Fourier Transform:

$$F ( \mu ) = \frac{1}{2\pi} \int_{-\infty}^\infty f(x) e^{i \mu x} dx$$

But clearly this is equivalent to 

\begin{equation}\label{split}
F ( \mu ) = \frac{1}{2\pi}\left( \int_{-\infty}^0 f(x) e^{i \mu x} dx+ \int_{0}^\infty f(x) e^{i \mu x} dx\right)
\end{equation}

Now we can plug in with our particular $f$

$$F ( \mu ) = \frac{1}{2\pi}\left( \int_{-\infty}^0 e^{2x} e^{i \mu x} dx+ \int_{0}^\infty e^{-3x} e^{i \mu x} dx\right)$$

$$= \frac{1}{2\pi}\left( \int_{-\infty}^0  e^{2x+ i \mu x} dx+ \int_{0}^\infty e^{i \mu x - 3x} dx\right)$$

Now we compute the antiderivative yielding

$$ \frac{1}{2\pi}\left( \left[  -\frac{i e^{x (2+i \mu )}}{-2 i+\mu } \right]_{-\infty}^0 + \left[ -\frac{i e^{i x (3 i+\mu )}}{3 i+\mu } \right]_{0}^\infty \right)$$

$$ =\frac{1}{2\pi}\left( \left[  \frac{ e^{x (2+i \mu )}}{2 +i\mu } \right]_{-\infty}^0 + \left[ \frac{ e^{i x (3 i+\mu )}}{-3 +i\mu } \right]_{0}^\infty \right)$$


Now, since 

$$\lim_{x \to -\infty}{ \frac{ e^{x (2+i \mu )}}{2 +i\mu }}=0=\lim_{x \to \infty}{ \frac{ e^{i x (3 i+\mu )}}{-3 +i\mu }}$$

we evaluate at the integration limits to get

$$ \frac{1}{2\pi}\left(   \frac{ 1}{2 +i\mu }  - \frac{ 1}{-3 +i\mu }  \right)$$

$$=\frac{5}{2 \pi  \left(6+i \mu +\mu ^2\right)}$$

So, the Fourier Transform of the given function is

$$F(\mu)=\frac{5}{2 \pi  \left(6+i \mu +\mu ^2\right)}$$


\subsection{Problem 5}

$$f(x)=\cos{x}e^{-|x|}$$

So, by Equation \ref{split} the Fourier Transform of this function is

$$F ( \mu ) = \frac{1}{2\pi}\left( \int_{-\infty}^0 f(x) e^{i \mu x} dx+ \int_{0}^\infty f(x) e^{i \mu x} dx\right)$$

$$= \frac{1}{2\pi}\left( \int_{-\infty}^0 \cos{x}e^{-|x|} e^{i \mu x} dx+ \int_{0}^\infty \cos{x}e^{-|x|} e^{i \mu x} dx\right)$$

Since in each of these ranges of integration we are guaranteed $x<0, x>0$ respectively we may replace $|x|$ with $-x, x$ respectively yielding

$$\frac{1}{2\pi}\left( \int_{-\infty}^0 \cos{x}e^{x} e^{i \mu x} dx+ \int_{0}^\infty \cos{x}e^{-x} e^{i \mu x} dx\right)$$

$$= \frac{1}{2\pi}\left( \int_{-\infty}^0 \cos{x}e^{i \mu x+x} dx+ \int_{0}^\infty \cos{x} e^{i \mu x-x} dx\right)$$

Integrating this expression we get 

$$= \frac{1}{2\pi}(  \left[ -\frac{i e^{x+i x \mu } }{-2-2 i \mu +\mu ^2}((-i+\mu ) \cos{x}-i \sin{x}) \right]_{-\infty}^0$$
$$+ \left[ \frac{e^{i x (i+\mu )}}{-2+2 i \mu +\mu ^2} (\cos{x}-i \mu  \cos{x}-\sin{x})  \right]_{0}^\infty  )$$

As before we observe that the $lim_{x \to \infty}$, $\lim_{x \to -\infty}$ of the respective expressions is $0$. Therefore, evaluating at the limits of integration we get

$$\frac{1}{2\pi} \left(   -\frac{i}{-2-2 i \mu +\mu ^2}((-i+\mu )) - \frac{1}{-2+2 i \mu +\mu ^2} (1-i \mu )    \right)$$

$$=\frac{2+\mu ^2}{4 \pi +\pi  \mu ^4}$$

So we have computed the Fourier Transform of the given function

\[F(\mu)=\frac{2+\mu ^2}{4 \pi +\pi  \mu ^4}\]

\subsection{Problem 7}

\[f(x)=\frac{2x}{{(1+x^2)}^2}\]

We observe that it is very difficult to compute the integral in the definition of the Fourier Transform for this $x$. However, if we realize that

\[\frac{d}{dx}\frac{-1}{(-1)^2+x^2}=f(x)\]

and employ the relationship between the Fourier Transform of a function and that of its derivative outlined in \cite[Page 280]{pinsky} then this problem becomes much easier. 

Call 

\[g(x)=\frac{-1}{(-1)^2+x^2}\]

Since $g(x)$ is of the form $g(x)=a/(a^2+x^2)$ its Fourier Transform is given by $G(\mu)=\frac{1}{2}e^{-a|\mu|}=\frac{1}{2}e^{|\mu|}$\cite[Page 285]{pinsky}. Moreover we have by the aforementioned relationship $\frac{d}{dx}g(x)=f(x) \Rightarrow i\mu G(\mu)=F(\mu)$ where $F(\mu)$ is the Fourier Transform of $f(x)$.

Thus, the Fourier Transform of the given function is

\[F(\mu)=\frac{i\mu}{2}e^{-|\mu|} \]

\subsection{Problem 9}

\[f(x)=\cos{x}e^{-x^2/2}\]

Again this problem can be greatly simplified by expressing $f(x)$ in terms of other functions whose Fourier Transforms are known. First let us replace Cosine by its definition in terms of complex powers of e, that is

\[f(x)=\frac{1}{2}\left(e^{i x}+e^{-i x}\right)e^{-x^2/2}\]

Now, write 

\[g(x)=e^{-x^2/2}\]

\[f(x)=\frac{1}{2}\left(e^{i x}+e^{-i x}\right)g(x)\]

\[=\frac{1}{2}e^{i x}g(x)+\frac{1}{2}e^{-i x}g(x)\]

but now, we can employ linearity, phase factor \cite[Page 280]{pinsky} to rewrite the Fourier Transform as

\[\frac{1}{2}G(\mu+1)+\frac{1}{2}G(\mu-1)\]

where $G(\mu)$ is the Fourier Transform of $g(x)$.

Now recall that $g(x)=e^{-x^2/2}$, and observe that this is just a multiple of a special case of the general Gaussian distribution whose Fourier Transform is worked out in detail \cite[Page 282]{pinsky} already. In particular we observe that the Fourier Transform $G(\mu)$ is given by

\[\sqrt{2\pi}G(\mu)=e^{-\mu^2/2}\]
\[\Rightarrow G(\mu)=\frac{e^{-\mu^2/2}}{\sqrt{2\pi}}\]


Thus the Fourier Transform of $f(x)$, $F(\mu)$ is given by,

\[F(\mu)=\frac{1}{2}G(\mu+1)+\frac{1}{2}G(\mu-1)\]

\[=\frac{1}{2}\frac{e^{-(\mu+1)^2/2}}{\sqrt{2\pi}}+\frac{1}{2}\frac{e^{-(\mu-1)^2/2}}{\sqrt{2\pi}}\]

\[=\frac{1}{2\sqrt{2\pi}}\left(e^{-\frac{1}{2}(\mu+1)^2}+e^{-\frac{1}{2}(\mu-1)^2}\right)\]



\begin{thebibliography}{9}

	\bibitem{pinsky}
	  Mark Pinsky,
	  \emph{Partial Differential Equations and Boundary Value Problems with Applications}.
	  Waveland Press, Illinois,
	  3rd Edition,
	  2003.

\end{thebibliography}


\end{document}
